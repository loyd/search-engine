\Introduction

В данной работе рассматриваются системы полнотекстового поиска, которые позволяют искать слова в большом наборе документов и сортируют результаты поиска по релевантности найденных документов запросу. Алгоритмы полнотекстового поиска относятся к числу важнейших среди алгоритмов коллективного разума.

Информационный поиск~--- это очень широкая область с долгой историей. В этой работе затрагиваются лишь немногие ключевые идеи, но тем не менее дающие значительные результаты. Хотя в центре внимания будут алгоритмы поиска и ранжирования, а не требования к инфраструктуре, необходимой для индексирования больших участков Всемирной паутины, созданная в данной работе система должна быть способной индексировать до миллиона страниц в сутки, сохраняя при этом относительно небольшое время поиска.

В данной работе рассматриваются все основные этапы: обход страниц, выделение содержимого, индексирование документов и непосредственно поиск по собранному индексу согласно различным критериям для ранжирования.

Целью данной работы является разработка и реализация информационной системы для сбора информации в сети Интернет и последующего поиска по ней.

В рамках работы должны быть решены следующие задачи:
\begin{enumerate}
  \item Анализ предметной области;
  \item Разработка индекса~--- базы данных для хранения информации;
  \item Разработка поискового робота~--- программы для сбора информации в сети Интернет;
  \item Разработка поисковика~--- программы для поиска релевантных запросу страниц в индексе;
  \item Разработка поисковой страницы~--- веб-приложения для осуществления запросов;
\end{enumerate}
